\chapter{Conclusions and Future Scope}\label{6}

This chapter provides a concise description of the achieved results in the previous chapters, answers to the formulated research questions in Chapter \ref{1} and the future scope of this thesis is presented at the end.

\section{Summary}
The thesis tackles the research problems identified in Chapter \ref{1} related to the need for large scale offshore networks with the available technology and provides solutions to mitigate technical challenges related to voltage and frequency control while developing large scale offshore networks.

From literature, it is seen that the current state-of-the-art technology, \gls{MMC}-\gls{HVDC} transmission, for transferring offshore wind power is limited to a maximum capacity of 1.2 GW \cite{peralta2012detailed}. Hence, a configuration for a large scale network (greater than or equal 2 GW) with the available technology is currently lacking. It is expected that the upcoming \gls{OWF} projects will be utilizing 66 kV \gls{HVAC} transmission from \gls{OWF}s to the offshore converter station due to their significant advantages over the conventional combination of 33 kV and 145 kV \cite{dnv66kv}. Moreover, the technical challenges related to voltage and frequency control in islanding of \gls{OWF}s is not full filled in real-time application and lacks the major requirement of reactive current injection by the \gls{WG}s during dynamic conditions \cite{sethi_real-time_nodate-new}.

The thesis adapts the latest trend in technology and overcomes the above mentioned research gaps by achieving the overall goal of developing a generic digital twin model of 2 GW, 66 kV \gls{HVAC} offshore network in RSCAD and performing analysis on the dynamics of voltage and power at various locations in the network. The digital twin model developed in this thesis adapts a hybrid configuration with two \gls{MMC}s working in parallel and connecting four \gls{OWF}s to the onshore system. The chosen configuration is a modified layout of the hub-and-spoke in \cite{abb_hvdc_2018} and such a network is currently unavailable in existing literature. Additionally, the technical challenges faced while using conventional current control strategies in \gls{WG}s related to voltage and frequency control are identified in the literature and are mitigated by implementing a new control strategy, \gls{DVC}, in all the four \gls{WG}s in RSCAD for real-time application. 

To achieve the overall goal, initially, the \gls{DVC} is implemented for real-time application in RSCAD for one set of \gls{OWF} connected to a 66 kV \gls{HVAC} network. The large scale 2 GW offshore network is developed by modularly connecting four sets of the aforementioned single \gls{OWF} model in parallel. The 2 GW offshore wind power is transferred to the onshore network through two parallel connected \gls{MMC}s. The key features of the model include- all the \gls{WG}s are equipped with \gls{DVC}, one \gls{MMC} working in grid forming control and the other working in grid following control. Such a model is unique because it incorporates a hybrid layout that allows for the coordination of the above mentioned strategies during steady state and dynamic conditions in the network.    


Few significant results of the model include the achievement of short-term voltage stability in the network during severe dynamic scenarios, the requirement of reactive current injection by \gls{WG}s during islanding by performing three-phase fault analysis. These results confirm the stable operation of the network under severe dynamic scenarios and the corresponding controller models can be utilized for further advancements on the size of the large scale network.  

% The voltage stability and power flow in the network during severe dynamic conditions such as disconnection of one set of \gls{OWF} and a three-phase line to ground fault in the middle of an \gls{HVAC} cable, The \gls{DVC} implemented in Type-4 \gls{WG}s    

% The compelling need to move towards \gls{RES} and the technical challenges faced due to the penetration of \gls{PE} converters is detailed in Chapter \ref{1} of the thesis. The relevant theoretical background required for the thesis work is presented in Chapter \ref{2}. 



% After recognizing the need for a 66 kV \gls{HVAC} offshore transmission through literature, a 66 kV offshore network with Type-4 \gls{WG} is modelled for \gls{EMT} simulations in Chapter \ref{3}. The 66 kV \gls{HVAC} cables transfer power from the \gls{OWF} to the infinite grid. The conventional usage of an offshore collector platform where the 33 kV from \gls{OWF} is stepped up to nearly 145 kV is avoided when using 66 kV \gls{HVAC} transmission and direct usage of 66 kV allows for twice the amount of power to be transferred. To overcome the challenges faced by conventional current control, the identified \gls{DVC} is implemented in the \gls{GSC} of Type-4 \gls{WG} model for the network in RSCAD. The design of a \gls{HPF} at the output of \gls{GSC} to filter out the higher frequency components is depicted in Section \ref{HPF_design}. On running \gls{EMT} simulations, it is seen that \gls{DVC} provides effective voltage control and the required performance of reactive power injection during dynamic conditions as mentioned in major grid codes \cite{mohseni_review_2012} is achieved. The influence of the major element in \gls{DVC} i.e. the washout filters is detailed by understanding the effect of gain parameters. To validate the \gls{DVC} operation in RSCAD, the performance of the model is compared with the benchmark model in PowerFactory \cite{korai_dynamic_2019} for a similarly modelled 66 kV network. The working is tested for a severe disturbance such as a three-phase line to ground fault in the middle of the cable. It is observed that \gls{DVC} modelled in RSCAD provides similar characteristics as the benchmark \gls{DVC} model in PowerFactory. Thereby, a Type-4 \gls{WG} model with implemented \gls{DVC} in RSCAD working in connection with an infinite grid is successfully developed and validated. The model can be used as reference for future works due to its real-time application.  

% The effective method to accomplish the targets set for several European and other countries towards energy transition is to further the usage of large scale offshore wind energy. In this thesis, a large scale offshore network with \gls{DVC} in \gls{WG}s is developed as detailed in Chapter \ref{4}. This is done by modelling a 2 GW offshore network with four sets of \gls{WG}s (with \gls{DVC}s) connected to two \gls{MMC}s using \gls{HVAC} cables through a modular approach in RSCAD. Each \gls{WG} can be scaled up to have a maximum rating of 500 MW. The model also avoids the usage of a offshore collector platofrm by using 66 kV \gls{HVAC} transmission. The chosen topology for the network and the representation of subsystems using Tline modules in RSCAD is represented. The V/F control mode available in \cite{wachal2014guide} is developed for \gls{MMC}-1 suitable for a 66 kV network and the outer loop control of \gls{MMC}-2 is modified to be suitable for a 66 kV network. The work around to use an extra interface transformer in \gls{MMC}-2 bus due to the unavailability of 66 kV control structure for \gls{MMC} in RSCAD is explained. 


% The operation of the 2 GW model is detailed in Chapter \ref{5}. The sychronization of \gls{MMC}s, energization procedure of \gls{HVAC} cables and \gls{WG}s in real time environment are described. To achieve the overall goal defined in Chapter \ref{1}, the 2 GW test network is subjected to perturbations such as sudden disconnection of 1 set of \gls{WG} and three-phase fault in the middle of an \gls{HVAC} cable for dynamic analysis. A logic is developed in RSCAD for the operation of circuit breaker at the \gls{HVAC} cable end. 


% The performance of \gls{DVC} to work in islanding operation is depicted by the operation of the circuit breaker. The \gls{DVC} implemented in \gls{GSC} executes reactive current injection by \gls{WG} during islanding operation in a large scale network which is also a significant requirement for majority of the grid codes. A modular 2 GW model with an control strategy (\gls{DVC}) implemented in \gls{WG}, working in coordination with two \gls{MMC}s with different control strategies is successfully developed for \gls{EMT} simulations in the real-time environment.   


\section{Answers to Research Questions}
\vspace{2mm}
\paragraph{\textcolor{blue}{Research Question 1 : How effective is the \gls{DVC} when implemented in an \gls{EMT} average model of Type-4 \gls{WG} connected to a 66 kV equivalent \gls{HVAC} system in RSCAD?}}

\paragraph{} The modelling of \gls{DVC} in \gls{GSC} of Type-4 \gls{WG} connected to an infinite grid through 66 kV \gls{HVAC} cables in RSCAD is explained in Chapter \ref{3}. The performance of the implemented \gls{DVC} is tested for severe disturbances in the grid. The event three-phase line to ground short circuit in the middle of the \gls{HVAC} cable is chosen for analysis. The \gls{DVC} provides voltage control and effectively stabilizes the system during such severe disturbances when connected to a 66 kV infinite grid. In terms of reactive current injection during a three-phase line to ground fault in the middle of the cable, the \gls{DVC} shows the required performance by providing active current priority during steady state conditions and reactive current priority during dynamic conditions. Such a strategy provides control of voltage directly (when compared to indirect control of voltage in conventional current control) and avoids the need of an external control to be acted upon during dynamic conditions such as an FRT controller. 

\paragraph{\textcolor{blue}{Research Question 2 : What insights can be attained by the \gls{EMT} average model of Type-4 \gls{WG} with \gls{DVC} in 66 kV network in RSCAD in comparison with a similar system modelled with a simplified Type-4 \gls{WG} configuration with \gls{DVC} in DIgSILENT PowerFactory?}}

\paragraph{} The performance of the implemented \gls{DVC} in a average model Type-4 \gls{WG} in RSCAD for a 66 kV network is compared with the benchmark \gls{DVC} for a simplified Type-4 \gls{WG} model in PowerFactory for a similar 66 kV network in terms of short-term voltage stability and reactive current injection during severe disturbances in the network. A three-phase fault event in the middle of the \gls{HVAC} cable is simulated in both the models in \gls{EMT} platforms. The voltage and currents profiles are compared and found to be similar for both the models following the three-phase line to ground fault. This shows the validation of the \gls{DVC} implemented in the average model of the Type-4 \gls{WG} in RSCAD and hence can be further utilized for future research works. The results obtained for the 66 kV \gls{HVAC} network are in line with the findings from previous studies with \gls{DVC} for \gls{WG}s connected to 33 kV \gls{HVAC} network as in \cite{korai_dynamic_2019} and \cite{sethi_real-time_nodate-new}. The usage of the RSCAD model also allows for real time environment application.    

\paragraph{\textcolor{blue}{Research Question 3 : How can Type-4 \gls{WG}s with implemented \gls{DVC} work in coordination with offshore \gls{MMC}s within a multi-gigawatt offshore transmission network?}}

\paragraph{} Four sets of \gls{OWF}s are connected in parallel to a common bus to which the two \gls{MMC}s are connected. In the steady-state operation, the grid forming \gls{MMC}-1 provides the voltage reference, \gls{MMC}-2 and \gls{DVC} follow this reference. During the time of fault, the voltage reference at the \gls{PCC} is provided by the \gls{DVC} in \gls{WG}s. The coordination between the V/F control in \gls{MMC}-1, active power control in \gls{MMC}-2 and the \gls{DVC} in the \gls{OWF}s provide a synchronized operation during the steady state and dynamic conditions in the network. The \gls{DVC} in this case also contributes to the reactive current injection in case of islanding. The operation of the network remains stable during severe perturbations such as disconnection of one set of \gls{OWF} and three-phase line to ground fault in the middle of any \gls{HVAC} cable.

\paragraph{\textcolor{blue}{Research Question 4 : How effectively do Type-4 \gls{WG}s with implemented \gls{DVC} perform when connected in parallel operation?}}

\paragraph{} The \gls{OWF}s with \gls{DVC} when working in parallel have similar profiles for the voltage and currents in steady-state and dynamic conditions. During the disconnection of one set of \gls{OWF}, the \gls{DVC}s in the rest of the \gls{OWF}s together contribute to the power flow in the network. Depending on the active power specified in the \gls{MMC} with grid following control (\gls{MMC}-2 in this case), the connected \gls{DVC}s contribute equally to this power if all the \gls{OWF}s generate the same amount of power. This provides a stabilized operation in terms of voltage and power flow in the network during all conditions.  

\section{Recommendations for Future Work}
Having successfully achieved the defined goals for this thesis and based on the studies performed, the author provides recommendations in the following section:  
\begin{itemize}
    \item The availability of additional NovaCor processor at TU Delft could be fully utilized to scale up the model to a capacity of 4 GW by accordingly connecting \gls{WG}s and \gls{MMC}s in a modular approach. The practical scenario of multi-terminal network can be implemented and the performance of the \gls{WG}s with \gls{DVC} can be analyzed.
    \item The present model of \gls{DVC} is applicable for three-phase fault analysis. Analysis can be extended for single-phase faults that are more common to occur in the real scenario by developing a negative sequence control loop for the \gls{DVC}.
    \item After performing a techno-economic study, various other network topologies can be researched on and tested with the implemented \gls{DVC} to bring out the cost-benefit feature for the industry experts. 
    \item The integration of the onshore converter station to the system by removing the \gls{DC} sources from the present model can be accomplished and extensive fault analysis can be performed in the \gls{MMC}-\gls{HVDC} network.
    \item The practicality of the controller can be tested by performing Hardware In Loop testing of the implemented \gls{DVC} in a properly set-up test bench.
\end{itemize}