\chapter*{Abstract}
To meet the Paris agreement, large scale offshore wind energy deployment in the North Sea is a crucial requirement.  
This includes the possibility of developing offshore infrastructure for deploying offshore wind power generation with installed capacity ranging from 70 to 150 GW by 2040, and increasing upto 180 GW by 2045. Presently, the Voltage Source Converter based (VSC based) - High Voltage Direct Current (VSC-HVDC) transmission is considered the most suitable for transfer of offshore wind power from distant offshore wind farms (OWFs) to the onshore system. Amidst the available VSC topologies, Modular Multi-level Converter (MMC) topology is the most appropriate  solution for the transfer of offshore wind power to onshore systems due to their enhanced performance during offshore and onshore disturbances. However, the currently deployed state-of-the-art MMC-HVDC transmission has a maximum capacity of 1.2 GW. Hence, this demands for a generic model with parallel operation of MMC-HVDC transmission systems to transfer the bulk amount of power from large scale OWFs. 

Additionally, the implementation of large scale offshore networks leads to an increase in the penetration of power electronic (PE) converters in the electrical power system. The increase in PE converters causes technical challenges (e.g. due to unprecedented fast dynamic phenomena) related to voltage and frequency stability, and power flow coordination in the power system. In view of the OWFs, the currently available current injection based voltage control for PE converters are not suitable for voltage control in large scale PE dominated systems due to the absence of continuous voltage control and ineffectiveness during islanding. Moreover, in such power systems, it is not suitable for frequency control due to the absence of dynamic frequency control. Therefore, better control strategies are required in large scale offshore networks to enhance the dynamic characteristics of the power system. 

Conventionally, the OWFs are coupled to a AC collector platform through 33 kV High Voltage Alternating Current (HVAC) cables. The voltage is stepped-up to 145 kV at the collector platform and power is transferred to the offshore converter station using 145 kV HVAC cables. However, in the upcoming projects the rated voltage levels are expected to increase from 33 kV to 66 kV to avoid the use of such a collector platform and directly transfer power from OWFs to the offshore converter station using 66 kV HVAC cables.   

This thesis proposes a digital twin model of a 2 GW offshore network with the parallel operation of two MMC-HVDC transmission links connecting four OWFs to two onshore systems representing a large scale power system. The MMCs are connected to a common bus on the AC side of the network, with one MMC working in grid forming operation and the other in grid following operation. Additionally, to mitigate the challenges corresponding to voltage and frequency stability in large scale offshore networks, a Direct Voltage Control (DVC) strategy is implemented in the Type-4 wind generators (WGs) representing the OWFs. After analyzing the need for 66 kV HVAC transmission from the OWFs to the offshore converter station, a 66 kV offshore network is developed to achieve 2 GW offshore wind power transfer. The electrical power system is developed in the power system simulation software, RSCAD\textsuperscript{TM} Version 5.011.1, in order to perform Electro-Magnetic Transient (EMT) based simulations. 

Initially, a single OWF with DVC implemented in the WG connected to a AC equivalent system is modelled to test the performance of DVC in a 66 kV HVAC network. The DVC provides continuous voltage control that improves the dynamic performance of the power system. As mentioned in most of the grid codes, the significant requirement of reactive power injection by the OWF during dynamic conditions is satisfied by the controller. DVC also avoids the need of an external controller to perform such an action. To validate the working of the implemented DVC in RSCAD, a similar 66 kV HVAC network with the benchmark DVC model is developed in DIgSILENT PowerFactory\textsuperscript{TM} 2019 SP2 (x64), for EMT simulations and tested under severe dynamic conditions. Both the models provide similar results, confirming the validation of the RSCAD model. Moreover, the RSCAD model provides a better representation of the practical environment operation. 

To achieve the overall goal of developing a 2 GW offshore transmission network, a hybrid system with the "hub-and-spoke" principle is utilized in this thesis. The 2 GW offshore network is achieved by a modular approach of the single OWFs connected to two MMCs working in parallel. The coordination between the implemented DVC in WGs and the control structures in MMCs is analysed for different scenarios in the network. The performance of the 2 GW network in terms of short-term voltage stability and power flow during severe dynamic conditions in the grid is analyzed. The requirement of reactive power injection from an OWF during dynamic conditions is achieved by performing a three-phase fault scenario in the middle of the cable. 
